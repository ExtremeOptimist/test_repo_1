\documentclass[norsk,10pt,a4paper]{article}
\usepackage[utf8]{inputenc}
\usepackage[norsk]{babel}
\usepackage{amsmath}
\usepackage{amsfonts}
\usepackage{amssymb}
\usepackage{graphicx}
\usepackage{float}
\author{Bjørnar Størksen Wiik}
\title{Lab 4 protokoll}

%\usepackage{fancyhdr}
%\pagestyle{fancy}
%\fancyhead{}
%\fancyhead[]{Lab 2 del 2}
%\fancyfoot{}
%\fancyfoot[]{side av \thepage}
%\fancyfoot[]{Chapter \thechapter}


\begin{document}
	\tableofcontents
	
	\section{Hensikt}
	Lab 4 går ut på å bli kjent med ociloskop, og måling av vekselstrø kretser.
	
	\section{Utstyrsliste}
	Resistanser: 	0,6 W motstander av forskjellige verdier. \newline
	Voltmeter:		1 stk Fluke 179 true RMS DMM	\newline
	Amperemeter:	1 stk Fluke 179 true RMS DMM	\newline
	Ociloskop
	
	\section{1}
	\subsection{a}
	
	\subsection{b}
	\begin{tabular}{|c|c|c|c|}
		\hline 
		$R1=R2=1 k \Omega$  & amplitude UR2 teoretisk [V] & amplitude UR2 \ målt [V] & målefeil \% \\ 
		\hline 
		probe x1 & 2.5 & 2.57 & 2.8 \% \\ 
		\hline 
		probe x10 & 0.25 & 0.257 & 2.8 \% \\ 
		\hline 
	\end{tabular} 
	
	\begin{tabular}{|c|c|c|c|}
		\hline 
		$R1=R2=1 M \Omega$  & amplitude UR2 \ teoretisk [V] & amplitude UR2 \ målt [V] & målefeil \% \\ 
		\hline 
		probe x1 & 2.5 & 1.71 & 31.6\% \\ 
		\hline 
		probe x10 & 0.25 & 0.253 & 2.8 \% \\ 
		\hline 
	\end{tabular} 	
		
	\subsection{1d}
	
		\begin{figure} [H]
		\centering
		\includegraphics[width=1\linewidth]{Bilder_2/scope_1}
		\caption{Oppg 1c. Bilde av hva som skjer når man bytter til kanal 2, etter at
		trigging er korrekt for 1. Får forskjellige bilder av spenningen, hver gang}
		\label{fig:scope1}
	\end{figure}

	\begin{figure} [H]
		\centering
		\includegraphics[width=1\linewidth]{Bilder_2/scope_3}
		\caption{Oppg 1d. Bilde av at toppene skjer samtidig, når begge probene måler over 1 k ohm. Men spenning 2, er halvparten av 1.}
		\label{fig:scope3}
	\end{figure}
	
	\begin{figure} [H]
		\centering
		\includegraphics[width=1\linewidth]{Bilder_2/scope_4}
		\caption{Oppg 1d. Viser faseforskyvning, når man bytter en motstand med en kondensator.}
		\label{fig:scope4}
	\end{figure}
	
	\begin{figure} [H]
		\centering
		\includegraphics[width=1\linewidth]{Bilder_2/scope_5}
		\caption{Oppg 2a.
			Viser at vi måler +10, og minus 10 volt, og spenningen går som en rett linje.}
		\label{fig:scope5}
	\end{figure}
	
	Sjekk av figur \ref{fig:scope6}, roten av 2.
	kalkulert roten av 2 = 1.4142.
	kalkulert av graf:
	$\frac{makx}{acRMS} = \frac{2.93}{2} = 1.4650$
	
	\begin{figure} [H]
		\centering
		\includegraphics[width=1\linewidth]{Bilder_2/scope_6}
		\caption{Oppg 2b. viser amplitude og RMS.
			Frekvens 1 k hz, periode var 1 ms.}
		\label{fig:scope6}
	\end{figure}
	
	\begin{figure} [H]
		\centering
		\includegraphics[width=1\linewidth]{Bilder_2/scope_7}
		\caption{Oppg 2c. Bilde av firkant signal.}
		\label{fig:scope7}
	\end{figure}
	
	\begin{figure} [H]
		\centering
		\includegraphics[width=1\linewidth]{Bilder_2/scope_8}
		\caption{Oppg 2c. Trigge tid til firkantspenningen på signalgenerator inni osioloskop.}
		\label{fig:scope8}
	\end{figure}
	
	\begin{figure} [H]
		\centering
		\includegraphics[width=1\linewidth]{Bilder_2/scope_11}
		\caption{Oppg 2d. Viser at spennig med dc og ac, gir pulserende likespenning, 
			altså sinus kurve, men går aldri under 0, null ca. midt på skjermen. Multimeteret viste 2 VDC}
		\label{fig:scope11}
	\end{figure}
	
	\begin{figure} [H]
		\centering
		\includegraphics[width=1\linewidth]{Bilder_2/scope_12}
		\caption{Oppg 2e. viser p-p 5, volt, firkantspenning.}
		\label{fig:scope12}
	\end{figure}
	
	\begin{figure} [H]
		\centering
		\includegraphics[width=1\linewidth]{Bilder_2/scope_15}
		\caption{Oppg 3. viser beregnet tidskonstant, fra 10 til 6 v.}
		\label{fig:scope15}
	\end{figure}
	
	
\end{document}